\documentclass{article}
\usepackage{multicol}
\usepackage[margin=0.5in]{geometry}
\setlength\columnsep{60pt}
\begin{document}
\begin{multicols}{2}
Erant omnino itinera duo, quibus itineribus domo exire possent: unum per Sequanos, angustum et difficile, inter montem Iuram et flumen Rhodanum, vix qua singuli carri ducerentur, mons autem altissimus impendebat, ut facile perpauci prohibere possent; 2 alterum per provinciam nostram, multo facilius atque expeditius, propterea quod inter fines Helvetiorum et Allobrogum, qui nuper pacati erant, Rhodanus fluit isque non nullis locis vado transitur. 3 Extremum oppidum Allobrogum est proximumque Helvetiorum finibus Genava. Ex eo oppido pons ad Helvetios pertinet. Allobrogibus sese vel persuasuros, quod nondum bono animo in populum Romanum viderentur, existimabant vel vi coacturos ut per suos fines eos ire paterentur. Omnibus rebus ad profectionem comparatis diem dicunt, qua die ad ripam Rhodani omnes conveniant. Is dies erat a. d. V. Kal. Apr. L. Pisone, A. Gabinio consulibus.

\vfill\null
\columnbreak
There were in all two routes, by which they could go forth from the house of country one through the Sequani narrow and difficult, narrow and difficult, between Mount Jura and the river Rhone, by which scarcely one wagon at a time should be led away, a very high mountain overhanging, so it can be easily intercept them; 2 the other, through our Province, much easier and freer from obstacles, because of flows between the boundaries of the Helvetii and the Allobroges, who had lately been subdued, and the Rhone and is in some places crossed by a ford. 3 Allobrogi nearest to farthest town of Switzerland Geneva. From this town a bridge extends to the Helvetii. They should either persuade the Allobroges, which is not as yet well-affected toward the Roman people, or compel them by, and supposed that to allow them to pass through their territories. Having provided every thing for the expedition, they appoint a day, on which day they should all meet on the bank of the Rhone. It was the day. d. V. Kal. Apr. L. Piso consulship.
\end{multicols}
\begin{multicols}{2}
Post eius mortem nihilo minus Helvetii id quod constituerant facere conantur, ut e finibus suis exeant. 2 Ubi iam se ad eam rem paratos esse arbitrati sunt, oppida sua omnia, numero ad duodecim, vicos ad quadringentos, reliqua privata aedificia incendunt; 3 frumentum omne, praeter quod secum portaturi erant, comburunt, ut domum reditionis spe sublata paratiores ad omnia pericula subeunda essent; trium mensum molita cibaria sibi quemque domo efferre iubent. Persuadent Rauracis et Tulingis et Latobrigis finitimis, uti eodem usi consilio oppidis suis vicisque exustis una cum iis proficiscantur, Boiosque, qui trans Rhenum incoluerant et in agrum Noricum transierant Noreiamque oppugnabant, receptos ad se socios sibi adsciscunt. 

\vfill\null
\columnbreak
After his death, the Helvetii nevertheless attempt to do that which they do, to go forth from their territories. 2 When she has been submitted to it in point of being ready that they were prepared, to all their towns, in number about twelve-to their villages about four hundred-and to the private dwellings that remained; 3 up all the corn, except what they were to carry with, so that with no destroying the hope of a return home, they were the more ready for undergoing all dangers; three months worth of food each person to carry on the crowd. They persuade their neighbors, and the Latobrigi the Rauraci, and the Tulingi, to adopt the same plan, and set out with them after burning down their towns and villages, the Boii, who were attacking the Noreia the other side of the Rhine, and had crossed over into the territory of Noricum, and receiving to themselves as allies to admit each other.
\end{multicols}
\begin{multicols}{2}
Gallia est omnis divisa in partes tres, quarum unam incolunt Belgae, aliam Aquitani, tertiam qui ipsorum lingua Celtae, nostra Galli appellantur. 2 Hi omnes lingua, institutis, legibus inter se differunt. Gallos ab Aquitanis Garumna flumen, a Belgis Matrona et Sequana dividit. 3 Horum omnium fortissimi sunt Belgae, propterea quod a cultu atque humanitate provinciae longissime absunt, minimeque ad eos mercatores saepe commeant atque ea quae ad effeminandos animos pertinent important, 4 proximique sunt Germanis, qui trans Rhenum incolunt, quibuscum continenter bellum gerunt. Qua de causa Helvetii quoque reliquos Gallos virtute praecedunt, quod fere cotidianis proeliis cum Germanis contendunt, cum aut suis finibus eos prohibent aut ipsi in eorum finibus bellum gerunt. 5 [Eorum una, pars, quam Gallos obtinere dictum est, initium capit a flumine Rhodano, continetur Garumna flumine, Oceano, finibus Belgarum, attingit etiam ab Sequanis et Helvetiis flumen Rhenum, vergit ad septentriones. 6 Belgae ab extremis Galliae finibus oriuntur, pertinent ad inferiorem partem fluminis Rheni, spectant in septentrionem et orientem solem. 7 Aquitania a Garumna flumine ad Pyrenaeos montes et eam partem Oceani quae est ad Hispaniam pertinet; spectat inter occasum solis et septentriones.]

\vfill\null
\columnbreak
All Gaul is divided into three parts, one of which the Belgae inhabit, the Aquitani another, those who in their own language are called Celts, in our Gauls, the third. 2 All of the language, customs and laws. The river Garonne separates the Gauls from Iberians, the Marne and the Seine separate them from the Belgae. 3 Of all these, the Belgae are the bravest, because they are the farthest away from the culture and civilization of the Province, and the least frequently resort to them, and the merchants of the and import those things which tend to effeminate the mind;, 4 are the nearest to the Germans, who dwell beyond the Rhine, with whom they are continually waging war. It is for this reason the Helvetii also surpass the remaining Gauls in valor, as they contend with the Germans in almost daily battles, when they either keep them off his own territories, or themselves wage war on their frontiers. 5 [One part of these, it is said, than the Gauls occupy, takes its beginning at the river Rhone, the Garonne, the river, the ocean, and the territories of the Belgae; it borders, even the side of the Sequani and the Helvetii, upon the river Rhine, and stretches toward the north star. 6 The Belgae rises from the extreme frontier of Gaul, extend to the lower part of the river Rhine; and look toward the north and the east side of the sun. 7 from the Garonne river Rhine and the Atlantic to Spain; between west and north.]
\end{multicols}
\begin{multicols}{2}
Apud Helvetios longe nobilissimus fuit et ditissimus Orgetorix. Is M. Messala, [et P.] M. Pisone consulibus regni cupiditate inductus coniurationem nobilitatis fecit et civitati persuasit ut de finibus suis cum omnibus copiis exirent: 2 perfacile esse, cum virtute omnibus praestarent, totius Galliae imperio potiri. 3 Id hoc facilius iis persuasit, quod undique loci natura Helvetii continentur: una ex parte flumine Rheno latissimo atque altissimo, qui agrum Helvetium a Germanis dividit; altera ex parte monte Iura altissimo, qui est inter Sequanos et Helvetios; tertia lacu Lemanno et flumine Rhodano, qui provinciam nostram ab Helvetiis dividit. 4 His rebus fiebat ut et minus late vagarentur et minus facile finitimis bellum inferre possent; 5 qua ex parte homines bellandi cupidi magno dolore adficiebantur. 6 Pro multitudine autem hominum et pro gloria belli atque fortitudinis angustos se fines habere arbitrabantur, qui in longitudinem milia passuum CCXL, in latitudinem CLXXX patebant.

\vfill\null
\columnbreak
Among the Helvetii, and rich was by far the most noble, Orgetorix. He, when Marcus Messala, [and N.] and Marcus Piso were consuls, incited by lust of sovereignty, formed a conspiracy among the nobility, the city, persuaded the people to go forth from their territories with all his forces: 2 easy, since they excelled all in valor, of the whole of Gaul. 3 To this he the more easily persuaded them, because the Helvetii are confined on every side, the nature of the place: on the one side by the Rhine, a very broad and deep river, which separates the Helvetian territory from the Germans; that on the one side by the Jura, a very high mountain, which is between the Sequani and the Helvetii; third by the Lake of Geneva, and by the river Rhone, which separates our Province from the Helvetii. 4 In the present circumstances it resulted, that they could range less widely, and could less easily make war upon their neighbors; 5 where the men fond of great pain. 6 In place of the multitude of their population, and their renown for warfare and bravery, but narrow limits, they were thinking that, he who is in the length of 240 miles, with a span of 180 were exposed to.
\end{multicols}
\begin{multicols}{2}
Caesari cum id nuntiatum esset, eos per provincia nostram iter facere conari, maturat ab urbe proficisci et quam maximis potest itineribus in Galliam ulteriorem contendit et ad Genavam pervenit. 2 Provinciae toti quam maximum potest militum numerum imperat (erat omnino in Gallia ulteriore legio una), pontem, qui erat ad Genavam, iubet rescindi. 3 Ubi de eius aventu Helvetii certiores facti sunt, legatos ad eum mittunt nobilissimos civitatis, cuius legationis Nammeius et Verucloetius principem locum obtinebant, qui dicerent sibi esse in animo sine ullo maleficio iter per provinciam facere, propterea quod aliud iter haberent nullum: rogare ut eius voluntate id sibi facere liceat. Caesar, quod memoria tenebat L. Cassium consulem occisum exercitumque eius ab Helvetiis pulsum et sub iugum missum, concedendum non putabat; 4 neque homines inimico animo, data facultate per provinciam itineris faciundi, temperaturos ab iniuria et maleficio existimabat. 5 Tamen, ut spatium intercedere posset dum milites quos imperaverat convenirent, legatis respondit diem se ad deliberandum sumpturum: si quid vellent, ad Id. April. reverterentur. 

\vfill\null
\columnbreak
When it was reported to Caesar that they were attempting to make their route through our Province he hastens to set out from the city, and, by as great marches as he can to Further Gaul, and arrives at Geneva. Up as great a number as possible of soldiers he commands the whole of the province, 2 (there was in all only one legion in Further Gaul), over the bridge, which was at Geneva to be broken down. 3 When they treat of his arrival the Helvetii are apprized of fact, they sent ambassadors to him, as embassadors, the most illustrious of the city, of which embassy was Numeius and Verudoctius held the pride of place in possession of this camp, those who were saying to Him to be in the mind, without any harm to march through the province of it, because you had no other route: that they requested, in order to will be allowed to do. Caesar, because he kept in remembrance that Lucius Cassius, the consul, had been slain, and his army routed and made to pass under the yoke by the Helvetii, did not think that is to be granted; 4 that men of hostile disposition, if an opportunity of marching through the Province were given them, would abstain from outrage and mischief. 5 However, a period might intervene, until the soldiers whom he had ordered assemble, he replied to the ambassadors that he would take time to deliberate: if they wanted to return on the Ides. April. return.
\end{multicols}
\end{document}